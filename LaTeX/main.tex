\documentclass{ctexart}
\usepackage[UTF8]{ctex}
\usepackage{graphicx} % Required for inserting images
\usepackage{amsmath}
\usepackage{hyperref}
%注释

\title{example}
\author{terry huang}
\date{July 2025}

\begin{document}

\maketitle
\tableofcontents

\newpage
\section{Section 1}
itemize.
\subsection{subsection 1.1}
This is itemize example.
if 列举的项中有编码, use enumerate to replace itemize.
\begin{itemize}
    \item Item 1.
    \item Item 1. 
    \item Item 2.
    \item \ldots 
    \item Item n. 
\end{itemize}

\section{Section 2 : table}
This is Section 2.
Here is a table.
\begin{table}
    \caption{The number of Iterations}
    \centering

    \begin{tabular}{c|c}
    \hline
    iter1 & iter2\\
    \hline
    31 & 25 \\
    20 & 17 \\
    45 & 37 \\
    \hline
    \end{tabular}
    \label{tab:my_label}
\end{table}



\section{Section 3 : math}
No numbered section 3.
符号和公式.
\subsection{subsection 3.1}
同一行的公式:$\sum_{i=0}^{10}i$\\
另起一行的公式:$$\sum_{i=0}^{10}i$$

This is a numbered equation:
\begin{equation}
\sum_{i=0}^{10} i = 0 + 1 + 2 + \ldots + 10
\end{equation}
This is an unnumbered equation:
\begin{equation*}
\prod_{i=1}^5 i = 1 \times 2 \times 3 \times 4 \times 5
\end{equation*}
This is a multi-line aligned equation:
\begin{align}
\sum_{i=0}^{10} i &= 0 + 1 + 2 + \ldots + 10 \\
\prod_{i=1}^5 i &= 1 \times 2 \times 3 \times 4 \times 5 
\end{align}
This is a display formula:
\[
\int_{x=1}^{10}\frac{1}{x^2} 
\]
\[
\frac{d}{dy} y^2
\]
\[
\lim_{n\to \infty} \frac{1}{n}
10 \equiv 1 \text{ (mod 3)}
\sqrt{\frac{a}{b+c}}(b+a)
\]


\subsection{subsection 3.2}
Some names:$\alpha \beta \gamma \sigma \epsilon \Sigma \Gamma$

$\neq \geq \leq \approx \equiv \int \forall \exists \partial \mathcal \sim 4^{12} 4^12 C_{60}^{3} \frac{1}{2}$

\subsection{subsection 3.3}
matrix
$$
\begin{bmatrix}
0 & 0 & 1 \\ 
1 & 0 & 0 \\ 
\end{bmatrix}
\begin{pmatrix}
1 & 4 & 0 \\ 
2 & 5 & 8 \\
\end{pmatrix}
\begin{vmatrix}
1 & 4 & 0 \\ 
2 & 5 & 8 \\
\end{vmatrix}
$$

\section{section 4}
graphic
\end{document}
